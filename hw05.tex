\documentclass[12pt]{article}
\usepackage[left=1.5cm,right=1.5cm,top=2cm,bottom=2cm,bindingoffset=0cm]{geometry}
\usepackage[utf8x]{inputenc}
\usepackage[english,russian]{babel}
\usepackage{cmap}
\usepackage{amssymb}
\usepackage{amsmath}
\usepackage{url}
\usepackage{pifont}
\usepackage{tikz}
\usepackage{verbatim}
\usepackage[thmmarks, amsmath, thref]{ntheorem}

\theoremstyle{nonumberplain}
\theoremheaderfont{\itshape}
\theorembodyfont{\upshape}
\theoremseparator{.}
\theoremsymbol{\ensuremath{\square}}
\newtheorem{lemm}{\underline{\textbf{Lemm}}}
\newtheorem{proof}{\underline{\textbf{Proof}}}
\theoremsymbol{\ensuremath{\blacksquare}}
\newtheorem{solution}{\underline{\textbf{Solution}}}
\theoremseparator{. ---}
\theoremsymbol{\mbox{\texttt{;o)}}}
\newtheorem{varsol}{Solution (variant)}

\usetikzlibrary{shapes,arrows}
\usetikzlibrary{positioning,automata}
\tikzset{every state/.style={minimum size=0.2cm},
initial text={}
}


\newenvironment{myauto}[1][3]
{
  \begin{center}
    \begin{tikzpicture}[> = stealth,node distance=#1cm, on grid, very thick]
}
{
    \end{tikzpicture}
  \end{center}
}


\begin{document}
\begin{center} {\LARGE Формальные языки} \end{center}

\begin{center} \Large домашнее задание до 23:59 26.03 \end{center}
\bigskip

\begin{enumerate}
	\setcounter{enumi}{1}
	\item Приведите грамматику в нормальную форму Хомского:\\
		\texttt{S $\rightarrow$ R S | R}\\
   		\texttt{R $\rightarrow$ a S b | c R d | a b | c d | $\varepsilon$}\\
   		Терминалы: \texttt{a, b, c, d,} нетерминалы: \texttt{R, S,} стартовый нетерминал: \texttt{S,} пустая строка: $\varepsilon$.
	\begin{solution}\
		\begin{enumerate}
		\item Избавимся от неодиночных терминалов и длинных правил\\
		$\begin{cases}
			\texttt{S $\rightarrow$ R S | R}\\
   			\texttt{R $\rightarrow$ A U | C V | A B | C D | $\varepsilon$}\\
   			\texttt{A $\rightarrow$ a, B $\rightarrow$ b, C $\rightarrow$ c, D $\rightarrow$ d}\\
   			\texttt{U $\rightarrow$ S B, V $\rightarrow$ R D}
		\end{cases}$
		\item Добавим правила для $\varepsilon$-порождающих терминалов \texttt{S} и \texttt{R}\\
		$\begin{cases}
			\texttt{S $\rightarrow$ R S | R | $\varepsilon$}\\
   			\texttt{R $\rightarrow$ A U | C V | A B | C D | $\varepsilon$}\\
   			\texttt{A $\rightarrow$ a, B $\rightarrow$ b, C $\rightarrow$ c, D $\rightarrow$ d}\\
   			\texttt{U $\rightarrow$ S B | B, V $\rightarrow$ R D | D}
		\end{cases}$
		\item Уберём $\varepsilon$-правила\\
		$\begin{cases}
			\texttt{S $\rightarrow$ R S | R | $\varepsilon$}\\
   			\texttt{R $\rightarrow$ A U | C V | A B | C D}\\
   			\texttt{A $\rightarrow$ a, B $\rightarrow$ b, C $\rightarrow$ c, D $\rightarrow$ d}\\
   			\texttt{U $\rightarrow$ S B | B, V $\rightarrow$ R D | D}
		\end{cases}$
		\item Введём новый стартовый терминал\\
		$\begin{cases}
			\texttt{S$_0$ $\rightarrow$ S | $\varepsilon$}\\
			\texttt{S $\rightarrow$ R S | R}\\
   			\texttt{R $\rightarrow$ A U | C V | A B | C D}\\
   			\texttt{A $\rightarrow$ a, B $\rightarrow$ b, C $\rightarrow$ c, D $\rightarrow$ d}\\
   			\texttt{U $\rightarrow$ S B | B, V $\rightarrow$ R D | D}
		\end{cases}$
		\item Замкнём цепочки\\
		$\begin{cases}
			\texttt{S$_0$ $\rightarrow$ R S | A U | C V | A B | C D | $\varepsilon$}\\
			\texttt{S $\rightarrow$ R S | A U | C V | A B | C D}\\
   			\texttt{R $\rightarrow$ A U | C V | A B | C D}\\
   			\texttt{A $\rightarrow$ a, B $\rightarrow$ b, C $\rightarrow$ c, D $\rightarrow$ d}\\
   			\texttt{U $\rightarrow$ S B | b, V $\rightarrow$ R D | d}
		\end{cases}$
		\end{enumerate}
		Получили нормальную форму Хомского.
	\end{solution}
	\item Является ли следующий язык контекстно-свободным? Если является, привести КС грамматику, иначе -- доказать.
	\begin{center}
		\{$a^m b^n$ $|$ $m + n > 0$, $(m + n)$ делится на 2\}
	\end{center}
	\begin{solution}
		Этот язык является контекстно-свободным и задаётся КС грамматикой
		\begin{center}
 		\texttt{S $\rightarrow$ a a S | S b b | a S b | a a | b b | a b}
 		\end{center}
 		Докажем по индукции по длине слова $k$, что все строки, которые описывает эта грамматика содержатся в языке. \underline{База:} k = 2. строки \texttt{a a,} \texttt{b b} и \texttt{a b} лежат в языке. \underline{Переход:} $k = m + n \rightarrow k = m + n + 2$. У нас есть слово из языка $a^m b^n$, такое что $k = m + n > 0$ и $k = (m + n)$ делится на 2. Мы можем приписать к нему слева \texttt{a a,} справа \texttt{b b}, или слева \texttt{a}, а справа \texttt{b,} то есть мы можем получить либо $a^{m + 2} b^n$, либо $a^m b^{n + 2}$, либо $a^{m + 1} b^{n + 1}$, но все эти слова лежат в нашем языке, так как $k + 2 = m + n + 2 > 0$ и делится на 2.\\\\
 		Теперь докажем, что грамматика содержит все слова языка. Допустим мы хотим получить слово $a^m b^n$, такое что $m + n > 0$ и $(m + n)$ делится на 2. Если $m = 0$, тогда $n$ чётное и мы можем взять и добавить нужное число раз \texttt{b b} справа и заменим \texttt{S} на \texttt{b b}. Если же $n = 0$, то $m$ чётное и мы возьмём и добавим нужное число раз \texttt{a a} слева и заменим \texttt{S} на \texttt{a a}. Если $m > 0, n > 0$ и $m$ чётное, тогда так как $m + n$ чётное, $n$ тоже чётное, тогда мы возьмём и добавим нужное число раз \texttt{a a} слева и \texttt{b b} справа заменим \texttt{S} на \texttt{a a}, например (в зависимости от того сколько каких букв добавили). И наконец, если $m > 0, n > 0$ и $m$ нечётное, тогда так как $m + n$ чётное, $n$ тоже нечётное, тогда мы возьмём \texttt{a b} и добавим нужное число раз \texttt{a a} слева и \texttt{b b} справа (так чтобы букв каждого типа было на одну меньше, чем нужно), а потом добавим \texttt{a} слева и \texttt{b} справа и заменим \texttt{S}.\\\\
 		Таким образом, грамматика содержит все слова языка и язык содержит все слова, которые описывает грамматика, значит, эта грамматика описывает язык.
	\end{solution}
\end{enumerate}


\end{document}