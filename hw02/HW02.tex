\documentclass[12pt]{article}
\usepackage[left=2cm,right=2cm,top=2cm,bottom=2cm,bindingoffset=0cm]{geometry}
\usepackage[utf8x]{inputenc}
\usepackage[english,russian]{babel}
\usepackage{cmap}
\usepackage{amssymb}
\usepackage{amsmath}
\usepackage{url}
\usepackage{pifont}
\usepackage{tikz}
\usepackage{verbatim}
\usepackage[thmmarks, amsmath, thref]{ntheorem}

\theoremstyle{nonumberplain}
\theoremheaderfont{\itshape}
\theorembodyfont{\upshape}
\theoremseparator{.}
\theoremsymbol{\ensuremath{\square}}
\newtheorem{lemm}{\underline{\textbf{Lemm}}}
\newtheorem{proof}{\underline{\textbf{Proof}}}
\theoremsymbol{\ensuremath{\blacksquare}}
\newtheorem{solution}{\underline{\textbf{Solution}}}
\theoremseparator{. ---}
\theoremsymbol{\mbox{\texttt{;o)}}}
\newtheorem{varsol}{Solution (variant)}

\usetikzlibrary{shapes,arrows}
\usetikzlibrary{positioning,automata}
\tikzset{every state/.style={minimum size=0.2cm},
initial text={}
}


\newenvironment{myauto}[1][3]
{
  \begin{center}
    \begin{tikzpicture}[> = stealth,node distance=#1cm, on grid, very thick]
}
{
    \end{tikzpicture}
  \end{center}
}


\begin{document}
\begin{center} {\LARGE Формальные языки} \end{center}

\begin{center} \Large домашнее задание до 23:59 05.03 \end{center}
\bigskip

\begin{enumerate}
  \item Доказать или опровергнуть утверждение: произведение двух минимальных автоматов всегда дает минимальный автомат (рассмотреть случаи для пересечения, объединения и разности языков).
  \begin{solution}\
  \\\texttt{Ответ:} утверждение неверно.\\
  Рассмотрим автоматы:
  \begin{enumerate}
  \item 
  \begin{myauto}
  	\node[state,initial]		 (q_0) 				 	 {$q_0$};
  	\node[state,accepting]		 (q_1)    [right=of q_0] {$q_1$};
  
  \path[->] (q_0) edge  [bend left = 15] node [above] {$a$}           (q_1)
  			(q_1) edge  [bend left = 15] node [below] {$a$}			  (q_0);
  \end{myauto}
  \item 
  \begin{myauto}
  	\node[state,initial,accepting] (s_0) 				 {$s_0$};
  	\node[state]		 		   (s_1)  [right=of q_0] {$s_1$};
  
  \path[->] (q_0) edge  [bend left = 15] node [above] {$a$}           (s_1)
  			(q_1) edge  [bend left = 15] node [below] {$a$}			  (s_0);
  \end{myauto}
  В их произведении без недостижимых вершин будет всего две вершины: $(q_0, s_0)$ и $(q_1, s_1)$, но в случае их пересечения минимальный автомат будет состоять из одной ничего не допускающей вершины, а в случае объединения -- из одной вершины, допускающей все. (В обоих случаях из вершины в себя есть петля). Для опровержения утверждения в случае разности рассмотрим 2 автомата типа (a): их произведение -- автомат типа (a), а минимальный автомат, как и в случае с пересечением будет состоять из одной вершины, которая все не допускает.
  \end{enumerate}
  
  \end{solution}
  \item Для регулярного выражения:
   \[ (a \mid b)^+ (aa \mid bb \mid abab \mid baba)^* (a \mid b)^+\]
  Построить эквивалентные:
  \begin{enumerate}
    \item Недетерминированный конечный автомат
   	\begin{solution}
   	\begin{myauto}
		\node[state,initial]  			 (q_0)                 {$q_0$};
    	\node[state]         	   	     (q_1)  [right=of q_0] {$q_1$};
    	\node[state]          			 (q_2)  [right=of q_1] {$q_2$};
    	\node[state]         		     (q_3)  [above left=of q_2] {$q_3$};
    	\node[state]         	   	     (q_4)  [above=of q_3] {$q_4$};
    	\node[state]         		     (q_5)  [right=of q_4] {$q_5$};
    	\node[state]        		     (q_6)  [below left=of q_2] {$q_6$};
    	\node[state]          			 (q_7)  [below=of q_6] {$q_7$};
    	\node[state]           			 (q_8)  [right=of q_7] {$q_8$};
    	\node[state]           			 (q_9)  [right=of q_2] {$q_9$};
    	\node[state,accepting]           (q_10) [right=of q_9] {$q_{10}$};

    \path[->] (q_0) edge              	  node [above] {$a, b$}        (q_1)
              (q_1) edge [loop above] 	  node [above] {$a, b$}        ()
                    edge              	  node [above] {$\varepsilon$} (q_2)
              (q_2) edge [bend left = 15] node [left]  {$a$}    		   (q_3)
                    edge [bend left = 15] node [right] {$b$}   		   (q_6)
                    edge 				  node [above] {$\varepsilon$} (q_9)
              (q_3) edge [bend left = 15] node [right] {$a$}    	       (q_2)
                    edge           	      node [left]  {$b$}    		   (q_4)
              (q_4) edge 			    	  node [above] {$a$} 		   (q_5)
              (q_5) edge 				  node [right] {$b$} 		   (q_2)
              (q_6) edge [bend left = 15] node [left]  {$b$} 		   (q_2)
              		edge 				  node [left]  {$a$} 		   (q_7)
              (q_7) edge 				  node [below] {$b$} 		   (q_8)
              (q_8) edge 				  node [right] {$a$} 		   (q_2)
              (q_9) edge [loop above]	  node [above] {$a, b$} 		   ()
              		edge 				  node [above] {$a, b$} 		   (q_10);
	\end{myauto}
	\end{solution}
    \item Недетерминированный конечный автомат без $\varepsilon$-переходов
    \begin{solution}
    Заметим, что \[ (a \mid b)^+ (aa \mid bb \mid abab \mid baba)^* (a \mid b)^+\] это в точности \[ (a \mid b)^+ (a \mid b)^+,\] поэтому недетерминированный конечный автомат без $\varepsilon$-переходов выглядит так:
    \begin{myauto}
		\node[state,initial]  			 (q_0)                 {$q_0$};
    	\node[state]         	   	     (q_1)  [right=of q_0] {$q_1$};
    	\node[state,accepting]         	 (q_2)  [right=of q_1] {$q_2$};

    \path[->] (q_0) edge                  node [above] {$a, b$}        (q_1)
              (q_1) edge              	  node [above] {$a, b$}		   (q_2)
              		edge [loop above]	  node [above] {$a, b$}		   ()
              (q_2) edge [loop above]	  node [above] {$a, b$} 		   ();
	\end{myauto}
	\end{solution}
    \item Минимальный полный детерминированный конечный автомат
    \begin{solution}
		\begin{myauto}
		\node[state,initial]  			 (q_0)                 {$q_0$};
    	\node[state]         	   	     (q_1)  [right=of q_0] {$q_1$};
    	\node[state,accepting]         	 (q_2)  [right=of q_1] {$q_2$};

    \path[->] (q_0) edge                  node [above] {$a, b$}        (q_1)
              (q_1) edge              	  node [above] {$a, b$}		   (q_2)
              (q_2) edge [loop above]	  node [above] {$a, b$} 		   ();
		\end{myauto}
    \end{solution}
  \end{enumerate}
  \pagebreak
  \item Построить регулярное выражение, распознающее тот же язык, что и автомат:
  \begin{myauto}
    \node[state]           (q_2)                {$q_2$};
    \node[state,initial]   (q_0) [left=of  q_2] {$q_0$};
    \node[state]           (q_1) [above=of q_2] {$q_1$};
    \node[state]           (q_3) [below=of q_2] {$q_3$};
    \node[state,accepting] (q_4) [right=of q_2] {$q_4$};

    \path[->] (q_0) edge [loop above] node [above] {$a, b, c$} ()
                    edge              node [above] {$a$}       (q_1)
                    edge              node [above] {$b$}       (q_2)
                    edge              node [above] {$c$}       (q_3)
              (q_1) edge [loop above] node [above] {$b, c$}    ()
                    edge              node [above] {$a$}       (q_4)
              (q_2) edge [loop above] node [above] {$a, c$}    ()
                    edge              node [above] {$b$}       (q_4)
              (q_3) edge [loop above] node [above] {$a, b$}    ()
                    edge              node [above] {$c$}       (q_4)
    ;
  \end{myauto}
  \begin{solution}
  	\[ (a \mid b \mid c)^* ((a (b \mid c)^* a) \mid (b (a \mid c)^* b) \mid (c (a \mid b)^* c)) \]
  \end{solution}
  \begin{lemm}[обратная формулировка леммы о накачке]\ 
 	\\Если для языка $L$ над алфавитом $V$ имеет место:\\ $\displaystyle \forall n\in \mathbb {N}$ $\exists \alpha \in L\colon |\alpha |\geqslant n$, $\forall u,v,w\in V^{*}\colon (\alpha =uvw) \land (|v|\geqslant 1) \land (|uv|\leqslant n)$ $\exists i\in \mathbb {N} \cup \{0\} \colon$ $uv^{i}w\not \in L$, то язык $L$ -- неавтоматный.
  \end{lemm}
  \item Определить, является ли автоматным язык $\{ \omega \omega^r \mid \omega \in \{ 0, 1 \}^* \}$. Если является --- построить автомат, иначе --- доказать.
  \begin{solution}
  	Пусть $L$ -- язык из условия. Докажем с помощью леммы, что $L$ неавтоматный. По данному $n$ возьмем слово $\alpha = (10^n1)^2 \in L$. $(\alpha =uvw) \land (|v|\geqslant 1) \land (|uv|\leqslant n)$. Если $u = \varepsilon$, то $v = 10^{<n}$, т.е. в $v$ одна единица, тогда при $i = 2$: $|uv^iw|_1 = |vvw|_1 = 1 + 1 + 3 = 5$ -- нечетное $\Rightarrow uv^2w \not \in L$. А если $u \neq \varepsilon$, то $v = 0^k$, где $1 \leqslant k < n$, тогда при $i = 3$ размер первой половины $uv^iw$ равен $\frac{|uv^3w|}{2} = \frac{|uvw| + 2|v|}{2} = \frac{|\alpha|}{2} + |v|$. То есть после того как мы вставили 2 копии $v$ в $\alpha$, середина слова сдвинулась вправо на $|v| \geqslant 1$. Значит, вторая слева единица в $\alpha$ после этой операции оказалась в правой половине слова, тогда полученное слово $uv^3w \not \in L$, потому что в его левой и правой половинах не поровну единиц.
  \end{solution}
  \item Определить, является ли автоматным язык $\{ u a a v \mid u, v \in \{ a, b \}^* , |u|_b \geq |v|_a \}$. Если является --- построить автомат, иначе --- доказать.
  \begin{solution}
  	Пусть $L$ -- язык из условия. Докажем с помощью леммы, что $L$ неавтоматный. По данному $n$ возьмем слово $\alpha = b^naa(ba)^n \in L$. $(\alpha =uvw) \land (|v|\geqslant 1) \land (|uv|\leqslant n) \Rightarrow v = b^k$, где $k > 0$. При $i = 0$: $uv^iw = uw = b^{n - k}aa(ba)^n \not \in L$, так как $n - k < n$.
  \end{solution}
\end{enumerate}

\newpage

\begin{center}
  \Large{Пример применения алгоритма минимизации}
\end{center}

\bigskip

Минимизируем данный автомат:

\begin{center}
  \begin{tikzpicture}[> = stealth,node distance=3cm, on grid]
    \node[state]           (q_2)                      {C};
    \node[state,initial]   (q_0) [above left=of q_2]  {A};
    \node[state]           (q_1) [below left=of q_2]  {B};
    \node[state]           (q_3) [right=of q_2]       {D};
    \node[state]           (q_4) [above right=of q_3] {E};
    \node[state,accepting] (q_5) [below right=of q_3] {F};
    \node[state,accepting] (q_6) [above right=of q_5] {G};

    \path[->] (q_0) edge [bend left=15]  node [right] {$1$} (q_1)
                    edge                 node [above] {$0$} (q_2)
              (q_1) edge [bend left=15]  node [left]  {$1$} (q_0)
                    edge                 node [below] {$0$} (q_2)
              (q_2) edge [bend right=15] node [below] {$1$} (q_3)
                    edge [bend left=15]  node [above] {$0$} (q_3)
              (q_3) edge                 node [below] {$1$} (q_5)
                    edge                 node [above] {$0$} (q_4)
              (q_4) edge                 node [above] {$1$} (q_6)
                    edge                 node [right] {$0$} (q_5)
              (q_5) edge [loop below]    node         {$1$} ()
                    edge [loop left]     node         {$0$} ()
              (q_6) edge                 node [below] {$1$} (q_5)
                    edge [loop right]    node         {$0$} ();
  \end{tikzpicture}
\end{center}

Автомат полный, в нем нет недостижимых вершин --- продолжаем.

Строим обратное $\delta$ отображение.

\begin{tabular}{c|c|c}
$\delta^{-1}$ & 0 & 1 \\ \hline
A & --- & B \\
B & --- & A \\
C & A B & --- \\
D & C & C \\
E & D & --- \\
F & E F & D F G \\
G & G & E
\end{tabular}

Отмечаем в таблице и добавляем в очередь пары состояний, различаемых словом $\varepsilon$: все пары, один элемент которых --- терминальное состояние, а второй --- не терминальное состояние. Для данного автомата это пары

$(A, F), (B, F), (C, F), (D, F), (E,F), (A, G), (B, G), (C, G), (D, G), (E, G)$

Дальше итерируем процесс определения неэквивалентных состояний, пока очередь не оказывается пуста.

$(A, F)$ не дает нам новых неэквивалентных пар. Для $(B, F)$ находится 2 пары: $(A, D), (A, G)$. Первая пара не отмечена в таблице --- отмечаем и добавляем в очередь. Вторая пара уже отмечена в таблице, значит, ничего делать не надо. Переходим к следующей паре из очереди. Итерируем дальше, пока очередь не опустошится.

Результирующая таблица (заполнен только треугольник, потому что остальное симметрично) и порядок добавления пар в очередь.

\begin{tabular}{c|cc|cc|cc|c}
& A & B & C & D & E & F & G \\ \hline
A &&&&&&& \\
B &&&&&&& \\ \hline
C & \checkmark & \checkmark &&&&& \\
D & \checkmark & \checkmark & \checkmark &&&& \\ \hline
E & \checkmark & \checkmark & \checkmark & \checkmark &&& \\
F & \checkmark & \checkmark & \checkmark & \checkmark & \checkmark && \\ \hline
G & \checkmark & \checkmark & \checkmark & \checkmark & \checkmark && \\
\end{tabular}

Очередь:

$
(A, F), (B, F), (C, F), (D, F), (E,F), (A, G), (B, G), (C, G), (D, G), (E, G),
$

$
(B, D), (A, D), (A, E), (B, E), (C, E), (C, D), (D, E), (A,C), (B, C))
$

В таблице выделились классы эквивалентных вершин: $\{A, B\}, \{C\}, \{D\}, \{E\}, \{F,G\}$. Остается только нарисовать результирующий автомат с вершинами-классами. Переходы добавляются тогда, когда из какого-нибудь состояния первого класса есть переход в какое-нибудь состояние второго класса. Минимизированный автомат:

\begin{center}
  \begin{tikzpicture}[> = stealth,node distance=3cm, on grid]
    \node[state,initial]   (q_01)                     {AB};
    \node[state]           (q_2)  [right=of q_01]      {C};
    \node[state]           (q_3)  [right=of q_2]       {D};
    \node[state]           (q_4)  [above right=of q_3] {E};
    \node[state,accepting] (q_56) [below right=of q_3] {FG};

    \path[->] (q_01) edge [loop above]    node [above] {$1$} ()
                     edge                 node [above] {$0$} (q_2)
              (q_2)  edge [bend right=15] node [below] {$1$} (q_3)
                     edge [bend left=15]  node [above] {$0$} (q_3)
              (q_3)  edge                 node [below] {$1$} (q_56)
                     edge                 node [above] {$0$} (q_4)
              (q_4)  edge [bend right=15] node [left]  {$1$} (q_56)
                     edge [bend left=15]  node [right] {$0$} (q_56)
              (q_56) edge [loop below]    node         {$1$} ()
                     edge [loop left]     node         {$0$} ();
  \end{tikzpicture}
\end{center}

\end{document}
