\documentclass[12pt]{article}
\usepackage[left=1.5cm,right=1.5cm,top=2cm,bottom=2cm,bindingoffset=0cm]{geometry}
\usepackage[utf8x]{inputenc}
\usepackage[english,russian]{babel}
\usepackage{cmap}
\usepackage{amssymb}
\usepackage{amsmath}
\usepackage{url}
\usepackage{pifont}
\usepackage{tikz}
\usepackage{verbatim}
\usepackage[thmmarks, amsmath, thref]{ntheorem}

\theoremstyle{nonumberplain}
\theoremheaderfont{\itshape}
\theorembodyfont{\upshape}
\theoremseparator{.}
\theoremsymbol{\ensuremath{\square}}
\newtheorem{lemm}{\underline{\textbf{Lemm}}}
\newtheorem{proof}{\underline{\textbf{Proof}}}
\theoremsymbol{\ensuremath{\blacksquare}}
\newtheorem{solution}{\underline{\textbf{Solution}}}
\theoremseparator{. ---}
\theoremsymbol{\mbox{\texttt{;o)}}}
\newtheorem{varsol}{Solution (variant)}

\usetikzlibrary{shapes,arrows}
\usetikzlibrary{positioning,automata}
\tikzset{every state/.style={minimum size=0.2cm},
initial text={}
}


\newenvironment{myauto}[1][3]
{
  \begin{center}
    \begin{tikzpicture}[> = stealth,node distance=#1cm, on grid, very thick]
}
{
    \end{tikzpicture}
  \end{center}
}


\begin{document}
\begin{center} {\LARGE Формальные языки} \end{center}

\begin{center} \Large домашнее задание до 23:59 14.05 \end{center}
\bigskip

\begin{enumerate}
	\setcounter{enumi}{1}
	\item Построить однозначную КС грамматику, эквивалетную грамматике\\ $\texttt{S} \rightarrow\texttt{aSbbbb | aaaSbb | c}$.
	\begin{solution}\
		Заметим, что грамматика
		$\begin{cases}
			\texttt{S $\rightarrow$ aSbbbb | R}\\
   			\texttt{R $\rightarrow$ aaaSbb | c}\\
		\end{cases}$ эквивалентна исходной, так как результат не зависит от порядка применения правил. Рассмотрим слово $a^p c b^q$ из языка грамматики. Заметим, что $p = x + 3y$ и $q = 4x + 2y$, где $x$ и $y$ -- количества применений первого и второго правил соответственно. Эта система совместна и имеет единственное решение:
		 $\begin{cases}
			$$x = \frac{3q - 2p}{10}$$\\
   			$$y = \frac{4p - q}{10}$$\\
		\end{cases}$ Значит, грамматика однозначна.
	\end{solution}
	\item Описать язык, порождаемый грамматикой $\texttt{F} \rightarrow \varepsilon \texttt{ | aFaFbF}$
	\begin{solution}\ 
		Эта грамматика порождает язык, слова которого состоят только из букв \texttt{a} и \texttt{b}, причем всего букв \texttt{a} в слове ровно в 2 раза больше, чем букв \texttt{b} (следует из правила), и в любом префиксе слова букв \texttt{a} хотя бы в 2 раза больше, чем букв \texttt{b} (также следует из правила).
	\end{solution}
	\item Найти КС грамматику, порождающую пересечение языка, порождаемого грамматикой\\ $\texttt{F} \rightarrow \texttt{a | bF | cFF}$ с языком, порождаемым грамматикой
		$\begin{cases}
			\texttt{K $\rightarrow$ aM | cM}\\
			\texttt{M $\rightarrow$ aK | bK | $\varepsilon$}\\
		\end{cases}$
	\begin{solution}\ 
		Заметим, что в языке второй грамматики длина слов нечетная, потому что правила будут чередоваться, и мы закончим на применении второго. Также слова могут начинаться только на буквы \texttt{a} и \texttt{c}, и не может встретится подстрок \texttt{bb} и \texttt{cc}, то есть в слове из языка буквы \texttt{b} могут стоять только на чётных местах, а буквы \texttt{с} на нечётных. Преобразуем первую грамматику в соответствии с ограничениями:\\
		$\begin{cases}
			\texttt{EVEN$_b$ $\rightarrow$ b ODD$_c$}\\
			\texttt{ODD$_b$ $\rightarrow$ a | b EVEN$_c$}\\
			\texttt{EVEN$_c$ $\rightarrow$ c EVEN$_b$ ODD$_b$ | c ODD$_b$ EVEN$_c$}\\
			\texttt{ODD$_c$ $\rightarrow$ a | c EVEN$_b$ EVEN$_b$ | c ODD$_b$ ODD$_c$}\\
		\end{cases}$
	\end{solution}
\end{enumerate}


\end{document}